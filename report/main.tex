\documentclass{article}

% Language setting
% Replace `english' with e.g. `spanish' to change the document language
\usepackage[english]{babel}

% Set page size and margins
% Replace `letterpaper' with`a4paper' for UK/EU standard size
\usepackage[letterpaper,top=2cm,bottom=2cm,left=3cm,right=3cm,marginparwidth=1.75cm]{geometry}

% Useful packages
\usepackage{amsmath}
\usepackage{graphicx}
\usepackage[colorlinks=true, allcolors=blue]{hyperref}

\title{Student Performance Analysis and Prediction \\ Final project for the Foundations of Data Science course \\(a.a. 2023/2024) }
\author{Paolo Cursi 2155622, Tommaso Leonardi 1914546, \\Arianna Paolini 1943164, Stefano Saravalle 1948684, Pietro Signorino 2149741}

\begin{document}
\maketitle

\begin{abstract}
Academic success is relevant to students satisfaction and consequent commitment in studying, leading them to become well formed professional figures. This project aims to understand which are the factors that influence the most the scores that the students get in their exams, by considering the case of the British Open University. Linear and non-linear machine learning models have been trained to predict the performance of the students in various assessments.
\end{abstract}

\section{Introduction}
The Open University is a public university in England that provides data about demographic information and academic performance of its students in order to allow learning analytics. We based our project on \textit{The Open University Learning Analytics dataset}, which is also available on Kaggle (\url{https://www.kaggle.com/datasets/rocki37/open-university-learning-analytics-dataset}). \\

We were interested in this dataset as it provides data about students interaction with the \textit{Virtual Learning Environmment} (\textit{VLE}), an online platform holding teaching materials and resources for the university courses. This could be an opportunity to test how much new technologies can help the learning process. That, combined with the availability of information about the social status of the students (age, gender, disability, education level, etc.) and their academic career (scores in assessment, number of studied credits, etc.) makes the Open University dataset a good choice for our purposes. \\

We selected different machine learning models to implement the prediction of students scores in assessments, both in the form of a regression and a multinomial classification task (by dividing the students according to some thresholds on the score range): \textit{linear regression}, \textit{decision trees} and \textit{KNN regression} will be used for regression, \textit{logistic regression }and \textit{neural networks} for classification. Having multiple trained models for each task allow us to compare the results, determine which model performs best and try to understand why.

\section{Dataset}

\begin{figure}
\centering
\includegraphics[width=0.7\textwidth]{database.png}
\caption{\label{fig:dataset}The organization of the tables in the Open University Learning Analytics dataset (\url{https://analyse.kmi.open.ac.uk/open_dataset})}
\end{figure}

The Open University Learning Analytics dataset contains data about 22 courses with 32,593 registered students \cite{dataset}. It is organized in seven relational tables, as show in Figure \ref{fig:dataset}:
\begin{itemize}
    \item \textbf{courses}: includes the list of all available courses (which are called \textit{modules}) and their \textit{presentations} (i.e. the occurence of a course in a specific academic year). The columns are:
    \begin{itemize}
        \item \textit{code module}: code name of the module, which serves as the identifier;
        \item \textit{code presentation}: code name of the presentation, which consists of the year and “B” for the presentation starting in February or “J” for the presentation starting in October;
        \item \textit{length}: length of the module-presentation in days.
    \end{itemize}

    \item \textbf{assessments}: contains information about assessments in module-presentations, which are typically a number of assessments followed by the final exam. The columns are:
    \begin{itemize}
        \item \textit{code module}: identification code of the module to which the assessment belongs;
        \item \textit{code presentation}: identification code of the presentation to which the assessment belongs;
        \item \textit{id assessment}: identification number of the assessment;
        \item \textit{assessment type}: type of assessment, it can be: Tutor Marked Assessment (TMA), Computer Marked Assessment (CMA) or Final Exam (Exam);
        \item \textit{date}: information about the final submission date of the assessment calculated as the number of days since the start of the module-presentation (the starting date of the presentation has number 0);
        \item \textit{weight}: weight of the assessment in percentage: final exams are treated separately and have weight 100\%, the sum of all other assessments is 100\%.  
    \end{itemize}

    \item \textbf{vle}: includes data about the available materials in the VLE (html pages, pdf files, etc.); students interactions with the materials are recorded. The table contains the following columns:
    \begin{itemize}
        \item \textit{id site}: an identification number of the material;
        \item \textit{code module }: an identification code for module;
        \item \textit{code presentation}: an identification code of presentation;
        \item \textit{activity type}: the role associated with the module material;
        \item \textit{week from}: the week from which the material is planned to be used;
        \item \textit{week to}: week until which the material is planned to be used.
    \end{itemize}

    \item \textbf{studentInfo}: contains demographic information about the students together with their results. The columns are:
    \begin{itemize}
        \item \textit{code module}: an identification code for a module on which the student is registered;
        \item \textit{code presentation}: the identification code of the presentation during which the student is registered on the module;
        \item \textit{id student}: a unique identification number for the student;
        \item \textit{gender}: the student’s gender;
        \item \textit{region}: identifies the geographic region, where the student lived while taking the module-presentation;
        \item \textit{highest education}: highest student education level on entry to the module presentation;
        \item \textit{imd band}: specifies the \textit{Index of Multiple Depravation band} of the place where the student lived during the module-presentation;
        \item \textit{age band}: band of the student’s age;
        \item \textit{num. of prev. attempts }: the number times the student has attempted this module;
        \item \textit{studied credits}: the total number of credits for the modules the student is currently studying;
        \item \textit{disability}: indicates whether the student has declared a disability;
        \item \textit{final result}: student’s final result in the module-presentation.
    \end{itemize}

    \item \textbf{studentRegistration}: contains information about the time when the student registered or unregistered for the module presentation. The colums are:
    \begin{itemize}
        \item \textit{code module}: an identification code for a module;
        \item \textit{code presentation}: the identification code of the presentation;
        \item \textit{id student}: a unique identification number for the student;
        \item \textit{date registration}: the date of student’s registration on the module presentation, as the number of days measured relative to the start of the module-presentation (e.g. the negative value -30 means that the student registered to module presentation 30 days before it started);
        \item \textit{date unregistration}: date of student unregistration from the module presentation (students  who completed the course have this field empty; students who unregistered have withdrawal as the value of the \textit{final result} column in the \textit{studentInfo} table).
    \end{itemize}

    \item \textbf{studentAssessment}: shows the results of students’ assessments. If the student does not submit the assessment, no result is recorded. The final exam submissions is missing if the result of the assessments is not stored in the system. It contains the following columns:
    \begin{itemize}
        \item \textit{id assessment}: the identification number of the assessment;
        \item \textit{id student}: a unique identification number for the student;
        \item \textit{date submitted}: the date of student submission, measured as the number of days since the start of the module presentation;
        \item \textit{is banked}: a status flag indicating that the assessment result has been transferred from a previous presentation;
        \item \textit{score}: the student’s score in this assessment; the range is from 0 to 100. A score lower than 40 is interpreted as Fail.
    \end{itemize}

    \item \textbf{studentVle}: contains information about each student’s interactions with the materials in the VLE. The columns are:
    \begin{itemize}
        \item \textit{code module}: an identification code for a module;
        \item \textit{code presentation}: the identification code of the module presentation;
        \item \textit{id student}: a unique identification number for the student;
        \item \textit{id site}: an identification number for the VLE material;
        \item \textit{date}: the date of student’s interaction with the material measured as the number of days since the start of the module-presentation;
        \item \textit{sum click}: the number of times a student interacts with the material in that day.
    \end{itemize}

    \end{itemize}



\section{Data Preprocessing}
\section{ML models training and testing}
\subsection{Linear regression}
\subsection{Decision Trees}
\subsection{KNN Regression}
\subsection{Logistic regression}
\subsection{Neural Networks}
\section{Result analysis}

\bibliographystyle{alpha}
\bibliography{sample}

\end{document}